\documentclass[11pt]{article}
\usepackage{fullpage}
\usepackage{url}
\usepackage{longtable}

\newenvironment{tightlist}{\begin{list}{$\bullet$}{
  \setlength{\itemsep}{0mm}
  \setlength{\parsep}{0mm}
%  \setlength{\labelsep}{0mm}
%  \setlength{\labelwidth}{0mm}
%  \setlength{\topsep}{0mm}
}}{\end{list}}

\pagestyle{empty}

\begin{document}

\begin{center}
\begin{tabular}{|c|} \hline
First call for papers \\
\hline
\\ 
{\Large COORDINATION 2007} \\
9th International Conference on Coordination Models and Languages \\
Paphos, Cyprus 6 - 8 June 2007 \\
\\
\emph{New directions in Coordination} \\
\\ \hline
                Paper submission: 27 January 2007 \\
                Author notification: 7 March 2007 \\
                Camera-ready copy: 26 March 2007 \\ \hline
               \url{http://www.discotec07.cs.ucy.ac.cy} \\ \hline
\end{tabular}
\end{center}

\vspace*{.5in}

\noindent
Modern information systems rely increasingly on combining concurrent,
distributed, real-time, reconfigurable and heterogeneous components.
New models, architectures, languages, and verification techniques are
necessary to cope with the complexity induced by the demands of
today's software development.  COORDINATION aims to explore the
spectrum of languages, middleware, services, and algorithms that
separate behavior from interaction, therefore increasing modularity,
simplifying reasoning, and ultimately enhancing software development.

\vspace*{.1in}
\noindent
Topics of interest:

\begin{tightlist}
\item Programming language techniques that support orchestration and control
    of distributed and concurrent interaction.
\item Middleware architectures: shared spaces, publish-subscribe, event-based.
\item Dynamic software architectures: software composition and scripting
    languages, dynamic software evolution and update, configuration and
    deployment languages.
\item Dependable, Resource-aware, Real-time and Embedded system coordination.
\item Models and Foundations: component composition, verification, management
    of security and dynamic aspects of coordination.
\item Web services: Service-oriented Architectures, Workflow Systems.
\item Programming abstractions for decentralized distributed systems such as
    P2P, mobile ad-hoc and sensor networks.
\item Type systems and specification languages appropriate for coordination of
    concurrent systems.
\item Case studies from E-Commerce, Factory Automation, Collaboration, Command
    and Control, or other systems.
\end{tightlist}

\begin{center}
{\bf Program Committee}
\\

\begin{longtable}{ll}
  Nadia Busi &          University of Bologna, IT \\
  Vinny Cahill &        Trinity, IE \\
  Paolo Ciancarini &    University of Bologna, IT \\
  William Cook &        University of Texas, Austin, US \\
  John Field &          IBM, US \\
  Chris Gill &          Washington University, US \\
  Aniruddha Gokhale &   Vanderbilt, US \\
  Chris Hankin &        Imperial College, UK \\
  Mike Hicks &          University of Maryland, US \\
  Valerie Issarny &     INRIA, FR \\
  Christoph Kirsch &    University of Salzburg, AT \\
  Doug Lea &            SUNY Oswego, US \\
  Toby Lehman &         IBM, US \\
  Alberto Montresor &   University of Trento, IT \\
  Amy L. Murphy &      ITC-IRST, IT U. of Lugano, CH (Co-chair) \\
  Oscar Nierstrasz &    University of Bern, CH \\
  Anna Philippou &      University of Cyprus, GR \\
  Ernesto Pimentel &    University of Malaga, ES \\
  Giovanni Russello &   Imperial College, UK \\
  Jan Vitek &           Purdue University, US (Co-chair) \\
  Jim Waldo &           SUN Microsystems, US \\
  Herbert Wiklicky &    Imperial College, UK \\
\end{longtable}

{\bf Proceedings}
\end{center}

Proceedings of previous editions of this conference were published by
Springer, in the Lecture Notes in Computer Science (LNCS) series and
are available as LNCS volumes 1061, 1282, 1594, 1906, 2315, 2949, 3454
and 4038. Our intention is to continue this series.

Selected papers from COORDINATION will be invited to a special issue
of The Science of Computer Programming journal.

A best student paper award will be given at the conference. To be
eligible for consideration indicate on your submission if one or more
of the paper's authors are students.

\begin{center}
{\bf Submission Instructions}
\end{center}

Authors are invited to submit full papers electronically in PDF before
15 January 2007. Further instructions are available from the
conference web site.

Submissions must be formatted according to the LNCS guidelines (see
\url{http://www.springer.de/comp/lncs/authors.html}) and must not
exceed 17 pages in length (including all supplementary
material). Papers that are not in the requested format or exceed the
mandated length will be rejected without going through the review
process. Simultaneous or similar submissions to other conferences or
journals are not allowed.

\end{document}
